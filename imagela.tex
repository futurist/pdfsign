%!TEX TS-program = xelatex
%!TEX encoding = UTF-8 Unicode

% Install Tex Live Suite first: http://anorien.csc.warwick.ac.uk/mirrors/CTAN/systems/texlive/Images/texlive2015.iso
% Or follow the link: http://www.cnblogs.com/Z-X-L/articles/2636107.html
% fc-list :lang=zh-cn to view Installed Chinese fonts
%
% Command line :
%    xelatex "\def\WIDTH{600pt} \def\HEIGHT{600pt} \def\X{0pt} \def\Y{0pt} \def\W{550pt} \def\IMG{aaa.png} % Command line :
%    pdflatex "\def\WIDTH{100px} \def\HEIGHT{100px} \def\X{0px} \def\Y{0px} \def\W{50px} \def\IMG{aaa.png} % Command line :
%    pdflatex "\def\WIDTH{100px} \def\HEIGHT{100px} \def\X{0px} \def\Y{0px} \def\W{50px} \def\IMG{aaa.png} % Command line :
%    pdflatex "\def\WIDTH{100px} \def\HEIGHT{100px} \def\X{0px} \def\Y{0px} \def\W{50px} \def\IMG{aaa.png} \input{imagela.tex}"

\documentclass[oneside,final,12pt]{memoir}

\usepackage{calc}

\usepackage[
  paperwidth=\WIDTH,
  paperheight=\HEIGHT,
  margin=0px,
]{geometry}
\thispagestyle{empty} %% Remove header and footer.
\setlength{\marginparsep}{0px}
\setlength{\marginparwidth}{0px}
\setlength{\marginparpush}{0px}
\setlength{\parindent}{0px}
\setlength{\parskip}{0px}


\usepackage[absolute,overlay]{textpos}
\textblockorigin{0px}{0px}

\usepackage{pdfpages}
%\setlength{\pdfpxdimen}{1in/600} % Define resolution of PDF
\begin{document}


	\begin{textblock*}{\textwidth}(\X,\Y)
		\includegraphics[keepaspectratio=true,width=\W]{\IMG}%
	\end{textblock*}

	%\includepdf[keepaspectratio=true, pages=-, width=\textwidth, delta=0 0, offset=0 0px]{mongodb.pdf}


\end{document}
"

\documentclass[oneside,final,12pt]{memoir}

\usepackage{calc}

\usepackage[
  paperwidth=\WIDTH,
  paperheight=\HEIGHT,
  margin=0px,
]{geometry}
\thispagestyle{empty} %% Remove header and footer.
\setlength{\marginparsep}{0px}
\setlength{\marginparwidth}{0px}
\setlength{\marginparpush}{0px}
\setlength{\parindent}{0px}
\setlength{\parskip}{0px}


\usepackage[absolute,overlay]{textpos}
\textblockorigin{0px}{0px}

\usepackage{pdfpages}
%\setlength{\pdfpxdimen}{1in/600} % Define resolution of PDF
\begin{document}


	\begin{textblock*}{\textwidth}(\X,\Y)
		\includegraphics[keepaspectratio=true,width=\W]{\IMG}%
	\end{textblock*}

	%\includepdf[keepaspectratio=true, pages=-, width=\textwidth, delta=0 0, offset=0 0px]{mongodb.pdf}


\end{document}
"

\documentclass[oneside,final,12pt]{memoir}

\usepackage{calc}

\usepackage[
  paperwidth=\WIDTH,
  paperheight=\HEIGHT,
  margin=0px,
]{geometry}
\thispagestyle{empty} %% Remove header and footer.
\setlength{\marginparsep}{0px}
\setlength{\marginparwidth}{0px}
\setlength{\marginparpush}{0px}
\setlength{\parindent}{0px}
\setlength{\parskip}{0px}


\usepackage[absolute,overlay]{textpos}
\textblockorigin{0px}{0px}

\usepackage{pdfpages}
%\setlength{\pdfpxdimen}{1in/600} % Define resolution of PDF
\begin{document}


	\begin{textblock*}{\textwidth}(\X,\Y)
		\includegraphics[keepaspectratio=true,width=\W]{\IMG}%
	\end{textblock*}

	%\includepdf[keepaspectratio=true, pages=-, width=\textwidth, delta=0 0, offset=0 0px]{mongodb.pdf}


\end{document}
"

\documentclass[oneside,final,12pt]{article}

\usepackage{xeCJK}
\setmainfont{Arial}
\setromanfont{Times New Roman}
\setCJKmainfont{Microsoft YaHei}

\usepackage{calc}

\usepackage[
  paperwidth=\WIDTH,
  paperheight=\HEIGHT,
  margin=0pt,
]{geometry}
\thispagestyle{empty} %% Remove header and footer.
\setlength{\marginparsep}{0pt}
\setlength{\marginparwidth}{0pt}
\setlength{\marginparpush}{0pt}
\setlength{\parindent}{0pt}
\setlength{\parskip}{0pt}


\usepackage[absolute,overlay]{textpos}
\textblockorigin{0pt}{0pt}

\usepackage{pdfpages}
%\setlength{\pdfptdimen}{1in/600} % Define resolution of PDF
\begin{document}


%-----------------------xeCJK下设置中文字体------------------------------%  
\setCJKfamilyfont{song}{SimSun}                             %宋体 song  
\newcommand{\song}{\CJKfamily{song}}                        % 宋体   (Windows自带simsun.ttf)  
\setCJKfamilyfont{xs}{NSimSun}                              %新宋体 xs  
\newcommand{\xs}{\CJKfamily{xs}}  
\setCJKfamilyfont{fs}{Fangsong ti}                      %仿宋2312 fs  
\newcommand{\fs}{\CJKfamily{fs}}                            %仿宋体 (Windows自带simfs.ttf)  
\setCJKfamilyfont{kai}{KaiTi_GB2312}                        %楷体2312  kai  
\newcommand{\kai}{\CJKfamily{kai}}                            
\setCJKfamilyfont{yh}{Microsoft YaHei}                    %微软雅黑 yh  
\newcommand{\yh}{\CJKfamily{yh}}  
\setCJKfamilyfont{hei}{Adobe Heiti Std}                                    %黑体  hei  
\newcommand{\hei}{\CJKfamily{hei}}                          % 黑体   (Windows自带simhei.ttf)  
\setCJKfamilyfont{adobehei}{Adobe Heiti Std}                                    %黑体  hei  
\newcommand{\adobehei}{\CJKfamily{adobehei}}                          % 黑体   (Windows自带simhei.ttf)  

%------------------------------设置字体大小------------------------%  
\newcommand{\chuhao}{\fontsize{42pt}{\baselineskip}\selectfont}     %初号  
\newcommand{\xiaochuhao}{\fontsize{36pt}{\baselineskip}\selectfont} %小初号  
\newcommand{\yihao}{\fontsize{28pt}{\baselineskip}\selectfont}      %一号  
\newcommand{\erhao}{\fontsize{21pt}{\baselineskip}\selectfont}      %二号  
\newcommand{\xiaoerhao}{\fontsize{18pt}{\baselineskip}\selectfont}  %小二号  
\newcommand{\sanhao}{\fontsize{15.75pt}{\baselineskip}\selectfont}  %三号  
\newcommand{\sihao}{\fontsize{14pt}{\baselineskip}\selectfont}%     四号  
\newcommand{\xiaosihao}{\fontsize{12pt}{\baselineskip}\selectfont}  %小四号  
\newcommand{\wuhao}{\fontsize{10.5pt}{\baselineskip}\selectfont}    %五号  
\newcommand{\xiaowuhao}{\fontsize{9pt}{\baselineskip}\selectfont}   %小五号  
\newcommand{\liuhao}{\fontsize{7.875pt}{\baselineskip}\selectfont}  %六号  
\newcommand{\qihao}{\fontsize{5.25pt}{\baselineskip}\selectfont}    %七号  

oijsweogjoejgojowejg
23459872398572937
\song\chuhao 新技术革 \yh\wuhao \hei\yihao 命新技术革命世宽松
\sihao
新技术革命新的校
\qihao
oijsweogjoejgojowejg
23459872398572937

	\begin{textblock*}{\textwidth}(\X,\Y)
		\includegraphics[keepaspectratio=true,width=\W]{\IMG}%
	\end{textblock*}

	%\includepdf[keepaspectratio=true, pages=-, width=\textwidth, delta=0 0, offset=0 0pt]{mongodb.pdf}


\end{document}
